\section{Auf Transportstrecken}\label{sec:transport}

In den Zwischenräumen zwischen den Gerüsten sowie auf Kühlstrecken kühlt das Walzgut durch Wärmeabgabe an die Umgebung aus.
Diese Bereiche sollen hier als Transport bezeichnet werden.
Dabei treten 3 hauptsächliche Mechanismen des Wärmeübergangs auf: Kovektion, Wasserkühlung und Strahlung.
Die Temperaturänderung entlang der Transportstrecke ergibt sich also zu

\begin{equation}
    \TemperatureChange = \TemperatureChangeConvection + \TemperatureChangeWater + \TemperatureChangeRadiation
    \label{eq:DeltaT-transport}
\end{equation}

Die einzelnen Beiträge werden im folgenden erläutert.

\subsection{Konvektion}\label{subsec:transport-konvektion}

Wärmeübergang durch Konvektion bzw.\ durch direkten Kontakt mit der Atmosphäre wird über ein einfaches Wärmeübergangskoeffizientenkonzept modelliert.
Benötigte Eingabewerte sind der Koeffizient \TransferCoefficientConvection und die Umgebungstemperatur \EnvironmentTemperature.
Überschlägig kann $\TransferCoefficientConvection \approx \SI{15}{\watt\per\square\meter\per\kelvin}$ angenommen werden.
\ProfileCrossSection ist der Profilquerschnitt, \ProfilePerimeter dessen Umfang.
Die für den Transport benötigte Zeit \Time ist ein Eingabewert.

\begin{equation}
    \TemperatureChangeConvection = \frac{\TransferCoefficientConvection \ProfilePerimeter \left( \Temperature - \EnvironmentTemperature \right) \Time}{\ProfileCrossSection \MassDensity \HeatCapacity}
    \label{eq:DeltaTK}
\end{equation}

\subsection{Wasserkühlung}\label{subsec:transport-wasserkuehlung}

Die Wasserkühlung wird ebenso über einen Wärmeübergangskoeffizienten \TransferCoefficientWater modelliert, aber getrennt betrachtet da sich der Koeffizient von dem der Konvektion unterscheidet und die Wassertemperatur \WaterTemperature im allgmeinen ungleich \EnvironmentTemperature ist.

\begin{equation}
    \TemperatureChangeWater = \frac{\TransferCoefficientWater \ProfilePerimeter \left( \Temperature - \WaterTemperature \right) \Time}{\ProfileCrossSection \MassDensity \HeatCapacity}
    \label{eq:DeltaTW}
\end{equation}

\subsection{Strahlung}\label{subsec:transport-strahlung}

Das Stefan-Boltzmann-Strahlungsgesetz ist im Unterschied zu den anderen beiden Ansätzen nichtlinear.
Die Strahlungsleistungs ändert sich mit der vierten Potenz der Temperatur.
$\BoltzmannRadiationCoefficient = \SI{5.6704}{\watt\per\square\meter\per\kelvin^4}$ ist die Stefan-Boltzmann-Konstante, \RelativeBoltzmannRadiationCoefficient der relative Strahlungskoeffizient.

\begin{equation}
    \TemperatureChangeRadiation = \frac{\ProfilePerimeter \RelativeBoltzmannRadiationCoefficient \BoltzmannRadiationCoefficient \left( \Temperature^4 - \EnvironmentTemperature^4 \right)}{\ProfileCrossSection \MassDensity \HeatCapacity}
    \label{eq:DeltaTS}
\end{equation}