\section{Im Walzstich}\label{sec:im-walzstich}

Im Walzstich treten andere Wärmeübergangsmechanismen als auf Transportstrecken auf.
Konvektion und Strahlung sollen hier ob der kleinen freien Oberflächen, der geringeren Übergangskoeffizienten und der kurzen Zeiten des Durchtritts durch den Walzspalt als vernachlässigbar angesehen werden.
Dagegen tritt Wärmeübergang durch metallischen Kontakt mit den Walzen, sowie Wärmegenerierung durch Umformung auf.
Die Temperaturänderung entlang im Walzstich ergibt sich also zu

\begin{equation}
    \TemperatureChange = \TemperatureChangeDeformation + \TemperatureChangeContact
    \label{eq:DeltaT-stich}
\end{equation}

Die einzelnen Beiträge werden im folgenden erläutert.

\subsection{Umformung}\label{subsec:im-walzstich-umformung}

Die Umformleistung wird zum einen in der Mikrostruktur des Werkstoffes gespeichert, zum weit größeren Teil aber als Wärme dissipiert.
Überschlägig kann ein dissipierter Anteil von \SI{95}{\percent} angesetzt werden.
\FlowStress ist der Umformwiderstand, welcher zum einen von der Fließspannung des Werkstoffes, zum anderen von der Geometrie des Walzspaltes abhängig ist.
$\Delta\EquivalentStrain$ ist die Formänderung im Walzstich.

\begin{equation}
    \TemperatureChangeContact = \num{0.95} \frac{\FlowStress \Delta\EquivalentStrain}{\MassDensity \HeatCapacity}
    \label{eq:DeltaTU}
\end{equation}

\subsection{Kontakt}\label{subsec:im-walzstich-kontakt}

Über die gedrückte Fläche \ContactArea geht Wärme vom Walzgut in die Walzen mit der Temperatur \RollTemperature über.
Der Koeffizient \TransferCoefficientContact fällt hier weit höher als bei Konvektion aus, ugf. \SIrange{2000}{6000}{\watt\per\square\meter\per\kelvin}.
Das Volumen des Walzgutes im Walzspalt wird hierbei zu $\Volume = \ContactLength \left( \ProfileCrossSection_\IndexEntry + \ProfileCrossSection_\IndexExit \right) / 2$ angenommen.

\begin{equation}
    \TemperatureChangeContact = \frac{\TransferCoefficientContact \ContactArea \left( \Temperature - \RollTemperature \right) \Time}{\Volume \MassDensity \HeatCapacity}
    \label{eq:DeltaTC}
\end{equation}